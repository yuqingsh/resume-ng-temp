% !TeX TS-program = xelatex

\documentclass{resume}
\ResumeName{沈雨清}

% 如果想插入照片,请使用以下两个库。
% \usepackage{graphicx}
% \usepackage{tikz}

\begin{document}

\ResumeContacts{
  (+86)180-1311-5939,%
  \ResumeUrl{yuqingsh@andrew.cmu.edu}{yuqingsh@andrew.cmu.edu},%
  % \ResumeUrl{https://www.linkedin.com/in/yuqingsh/}{https://www.linkedin.com/in/yuqingsh/} \footnote{下划线内容包含超链接。},%
  \ResumeUrl{https://github.com/yuqingsh}{https://github.com/yuqingsh} \footnote{下划线内容包含超链接。},%
}

% 如果想插入照片,请取消此代码的注释。
% 但是默认不推荐插入照片,因为这不是简历的重点。
% 如果默认的照片插入格式不能满足你的需求,你可以尝试调整照片的大小,或者使用其他的插入照片的方法。
% 不然,也可以先渲染 PDF 简历,然后用其他工具在 PDF 上叠加照片。
% \begin{tikzpicture}[remember picture, overlay]
%   \node [anchor=north east, inner sep=1cm]  at (current page.north east)
%      {\includegraphics[width=2cm]{image.png}};
% \end{tikzpicture}

\ResumeTitle


\section{教育经历}
\ResumeItem
[卡耐基梅隆大学|硕士研究生]
{卡耐基梅隆大学}
[\textnormal{Heinz信息系统及公共政策学院|}  信息系统管理硕士]
[2022.08—2023.12(预计)]

\textbf{GPA: 3.59/4.0},主要研究方向为\textbf{深度强化学习},在分布式系统及深度学习领域方面有一定的研究和工程经验。\textbf{2024年应届生}。

% 主要研究成果为:发表在 XXX 期刊上的论文《XXX》。

\ResumeItem
[密歇根大学|本科生]
{密歇根大学}
[\textnormal{数学科学,文学理学与艺术学院|} 理学学士]
[2020.09—2022.05]

\textbf{GPA: 3.57/4.0},获得“James B. Angell Scholar”荣誉,以表彰连续在多个学期中取得全A的学术成就。

\section[技术能力]{技术能力\protect\footnote{与求职岗位无关的技能省略或用灰色表示。}}
\begin{itemize}
  \item \textbf{语言}: 编程不受特定语言限制。常用 Java, Python, C; 熟悉 Golang, C++;了解 \GrayText{JavaScript}, \GrayText{TypeScript}。
  \item \textbf{工作流}: Linux, Shell, (Neo)Vim, Git, GitHub, GitLab.
  \item \textbf{其他}: 有丰富的计算机视觉及自然语言处理的实践经验,熟悉 Pytorch 的使用。
\end{itemize}

\section{工作经历}

\ResumeItem{北京 ABCD 有限公司}
[后端开发实习生/XXXX]
[2020.10—2021.03]

\begin{itemize}
  \item \textbf{独立负责XXX业务后端的设计、开发、测试和部署。}通过 FaaS、Kafka 等平台实现站内信模板渲染服务。向上游提供 SDK 代码,增加或升级了多种离线和在线逻辑。完成了业务对站内信的多样需求。
  \item \textbf{参与 XXX 的需求分析,系统技术方案设计;完成需求开发、灰度测试、上线和监控。}
\end{itemize}

\ResumeItem{北京 ABCD 有限公司}
[后端开发实习生/XXXX]
[2020.10—2021.03]

\begin{itemize}
  \item \textbf{独立负责XXX业务后端的设计、开发、测试和部署。}通过 FaaS、Kafka 等平台实现站内信模板渲染服务。向上游提供 SDK 代码,增加或升级了多种离线和在线逻辑。完成了业务对站内信的多样需求。
  \item \textbf{参与 XXX 的需求分析,系统技术方案设计;完成需求开发、灰度测试、上线和监控。}
\end{itemize}

\section{项目经历}

\ResumeItem{\textbf{BusTub} 基于 C++ 的简易单机数据库}
[ \textnormal{CMU 15-445} 课程]
\begin{itemize}
  \item 实现了基于可扩展哈希表和LRU-K的内存池管理。实现了可并发的B+树,支持乐观加锁的读写操作。
  \item 采用火山模型实现了查询、修改、连接、聚合等查询执行器,对部分查询进行了改写与下推。
  \item 采用 2PL 进行并发控制,支持死锁处理、多种隔离级别、表锁和行锁。
  \item 对数据库系统有了基本的认识和实践。
\end{itemize}

\ResumeItem{\textbf{动态内存分配器} 开发}
[ \textnormal{CMU 15-213} 课程]
\begin{itemize}
  \item 使用 \textbf{C} 语言实现了一个动态内存分配器,包括 malloc、free 和 realloc 函数。
  \item 采用了分隔适应策略(segregated fit strategy)优化了内存使用效率,并减少了内存碎片。
  \item 利用双向链表实现了空闲块的管理,有效提高了内存分配和释放的速度,吞吐量从5000Kops/s提升到15000Kops/s。
  \item 通过测试和调试,确保分配器具有稳定的性能,并在各种负载下均能正常工作。
\end{itemize}

\ResumeItem[Bayer Crop Science]
%{\ResumeUrl{https://github.com/BITNP/BIThesis}
{\textbf{Bayer Crop Science} 供应链研究实践 - 人工智能原型} %}
[主要开发者]
[2023.08 — 今]

\begin{itemize}
  \item 项目机构:CMU Heinz 应用分析小组
  \item 合作单位:拜耳作物科学(Bayer Crop Science)- 农业产品领先提供商
  \item 根据客户要求,使用 \textbf{机器学习} 构建一个\textbf{自然语言处理}模型来帮助供应链决策者在业务问题中进行规划决策。
  \item 客户无需理解模型的结构,也无需定义模型的任何数据、参数或约束。只需要口头描述要回答的问题,系统将通过对话收集配置模型所需的所有信息。模型会自动执行,并为用户提供对原始问题的答案。
  \item 负责了模型训练环境的搭建;完成了怎样的结果。
\end{itemize}

\section{个人总结}

\begin{itemize}
  \item 本人乐观开朗、在校成绩优异、自驱能力强,具有良好的沟通能力和团队合作精神。
  \item 可以使用英语进行工作交流(六级成绩 XXX),平时有阅读英文书籍和口语练习的习惯。
  \item 有六年 Linux 使用经验,较为丰富的软件开发经验、开源项目贡献和维护经验。善于技术写作,持续关注互联网技术发展。
\end{itemize}


\end{document}
